Aufgrund der Schaltung in Abbildung \ref{fig:Schaltbild} und der Vernachlässigung des Erregerstroms $i_E$, ergibt sich das mathematische Modell der Nebenschlussmaschine mit den drei Systemgleichungen des Unwuchtsystems:

\begin{equation}
\begin{aligned}
&U = U_R + U_i + U_L \\
&U = R_A i_A + K_A \dot{\varphi} + L_A \frac{\diff{i_A}}{\diff{t}} \\
\frac{\diff{i_A}}{\diff{t}} &= \frac{1}{L_A} (U - K_A \dot{\varphi} - i_A R_A) = f_1(U, \dot{\varphi}, i_A)
\label{enq:Spannungen}
\end{aligned}
\end{equation}

Translatorisch:

\begin{equation}
\begin{aligned}
(m_1 + m_2) \ddot{s} - m_2 e(\ddot{\varphi} \sin{\varphi} + \ddot{\varphi^2} \cos{\varphi}) + d_t \dot{s} + c s &= 0 \\
\ddot{s} = \frac{1}{m_1 + m_2}[m_2 e(\ddot{\varphi} \sin{\varphi} + \ddot{\varphi^2} \cos{\varphi}) - d_t \dot{s} - c s] &= f_2(\varphi, \dot{\varphi}, \ddot{\varphi}, s, \dot{s}) \label{enq:Bewgltrans}
\end{aligned}
\end{equation}

Rotatorisch:

\begin{equation}
\begin{aligned}
m_2 e^2 \dot{\varphi} - m_2 e \sin{\varphi} (\ddot{s} + g) d_r \dot{\varphi} - M_A &= 0 \\
m_2 e^2 \dot{\varphi} - m_2 e \sin{\varphi} (\ddot{s} + g) d_r \dot{\varphi} - K_A i_A &= 0 \\
\ddot{\varphi} = \frac{1}{m_2 e^2} [m_2 e \sin{\varphi} (\ddot{s} + g) d_r \dot{\varphi} + K_A i_A] &= f_3(\varphi, \dot{\varphi}, \ddot{s}, i_A) \label{enq:Bewglrot}
\end{aligned}
\end{equation}

Die gegebenen mechanischen Systemparameter lauten dabei:

\begin{table}[!hbt]
	\centering
	
	\begin{tabular}{| l | l |}
		\hline
		Massen & $m_1 = 90 \kilogram \text{; } m_2 = 10 \kilogram$ \\
		\hline
		Federkonstante & $c = 1600 \frac{\newton}{\meter}$ \\
		\hline
		Dämpfungskonstanten & $d_t = 5 \frac{\newton\second}{\meter}$ \\
		\hline
		Rotationsarm & $e = 0.2 \meter$ \\
		\hline
		Erdbeschleunigung & $g = 9.81 \frac{\meter}{\second^2}$ \\
		\hline
	\end{tabular}
	\captionabove{Systemparameter des mechanischen Modells}
	\label{tab:SystemparameterME}
\end{table}

Um die Unwuchtkraft des Systems zu bestimmen, muss das 2. Newton'sche Axiom \ref{enq:2.Newton} angewendet werden:

\begin{equation}
	\begin{aligned}
		F = m \cdot a \\
		\label{enq:2.Newton}
	\end{aligned}
\end{equation}

Daraus ergibt sich, angepasst an das Unwuchsystem:

\begin{equation}
\begin{aligned}
	F_U &= -m_2 \cdot \underline{a} \\
	\text{mit } \underline{a} &= \begin{bmatrix} e \ddot{\varphi} \cos \varphi  - e \dot{\varphi^2} \sin\varphi \\
	-e \ddot{\varphi} \sin \varphi  - e \dot{\varphi^2} \cos \varphi \end{bmatrix} \text{ ergibt sich:} \\
	\underline{F_U} &= \begin{bmatrix} e \ddot{\varphi} \cos \varphi  - e \dot{\varphi^2} \sin\varphi \\
	-e \ddot{\varphi} \sin \varphi  - e \dot{\varphi^2} \cos \varphi \end{bmatrix} \\
	\label{enq:Unwuchtkraft}
\end{aligned}
\end{equation}

Das Unwuchtsystem kann modelliert nun wie folgt dargestellt werden:

%Folgesituationen
\begin{figure}[hbt]
	\begin{tikzpicture}
	
	%Zeichnung Rechteck
	\draw (4,0.5) rectangle (9,2.5) node[midway, align=center](Block){Unwuchtsystem}; 
	
	%Pfeile
	\begin{scope}
	\draw[->] (0,1.5) -- (4,1.5);
	\draw[->] (9,1.5) -- (13,1.5);
	\end{scope}
	
	%Parameterbeschriftungen
	\draw(0,2.25) node(uin) {$u = U$};
	\draw(13.75,2.25) node(yvek) {$\underline{y} = \left[\begin{array}{c} s \\\ \dot{\varphi} \\\ M_A \\\ F_U \end{array}\right]$};
	
	\end{tikzpicture}
	
	%Abbildungsunterschrift
	\caption{Unwuchtsystem mit Eingangsspannung und Ausgang (Kinematik, Kinetik)}
	\label{fig:Unwuchtsystem}
	
\end{figure}