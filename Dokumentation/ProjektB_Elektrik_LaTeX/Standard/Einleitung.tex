% Autor:

\label{Einleitung}
In diesem Bericht wird ein Unwuchtsystem im Hinblick auf dessen elektronische Komponente simulationstechnisch betrachtet. Dafür wird der integrierte Gleichstrommotor vereinfacht durch das Ersatzschaltbild einer Nebenschlussmaschine dargestellt. Anhand dieser Vereinfachung wird das System einer Simulationsstudie unterzogen. Hierbei beginnt man das Originalsystem, mittels Systemanalysen und Modellbildung, in Modelle oder Teilsysteme zu überführen. In denen leichter gearbeitet werden kann. So zum Beispiel das Mathematische Modell. In diesem werden die einzelnen Differentialgleichungen aufgestellt, die zur Untersuchung des System notwendig sind. Im gleichen Schritt wird das Modell analysiert und dessen Systemeigenschaften, wie Ausgangs- und Eingangsgrößen, erarbeitet. Im vorliegenden System sind die Ausgangsgrößen beispielsweise das Motormoment $M_A$, die Unwuchtkraft $F_U$, die Winkelgeschwindigkeit $\Omega$ und die Strecke $s$. Um diese weiter untersuchen zu können werden die entdeckten Eigenschaften und Systemgleichungen in Matlab implementiert, um dort die gewünschten Simulationen durchzuführen. 