%Fazit - Autor:
\label{Fazit}

Abschließend kann man sagen, dass es beim Handling mit Matlab einige Probleme aufgrund von fehlenden Kenntnissen und Erfahrungen gab. So war uns anfangs nicht bewusst, dass die Eingangsvariablen bei Matlab Simulink händisch mit einer Referenz auf die im Workspace deklarierten Variablen zu versehen sind. Nachdem wir das festgestellt und behoben haben, wurden die Plots zwar verändert, jedoch entsprachen sie nicht unseren Erwartungen. \\
Eine Überlegung war, die Eingangsvariable händisch innerhalb in Matlab Simulink zu verändern, sodass wir dann das richtige Ergebnis bezüglich der Graphen herausbekommen. Jedoch hätte dies nach unserer Bewertung das Ziel, den Sinn und damit das ganze Projektergebnis verfälscht, sodass wir uns darauf festgelegt haben diese Möglichkeit nicht in Betracht zu ziehen. \\
Das fehlen von Matlab-Packages war unser erster Verdacht. Nach einer kurzen Recherche, welche Packages benötigt werden und einem Vergleich mit den von uns heruntergeladenen Daten, konnten wir diese Möglichkeit ausschließen. \\
Auffällig war jedoch, als unser Programm auf einem anderen Laptop mit einer älteren Matlab Version das richtige Ergebnis hervorbrachte. Somit lag der nächste Verdacht nun auf der heruntergeladenen Matlab Version. \\
Während der Installation der älteren Matlab Version, setzten wir uns weiter mit unserem Programm und den Aufbau des Quellcodes im Workspace auseinander. Auch die Schaltung in Simulink wurde weiter analysiert. Wir konnten so feststellen, dass aufgrund der Historie in Bezug auf die Referenzen der Variablen, einiges von Matlab verfälscht wurde. So veränderte Simulink die Eingangsvariable $U$, jedoch wurde diese einmalig abgespeichert, sodass Veränderungen nicht übernommen wurden. Die Ursache hierfür liegt darin, dass die von Simulink erstellte Variable, jene aus dem Workspace überschreibt. \\
Daher mussten wir uns die vom model Explorer erstellten Variablen und deren Referenzen genauer anschauen. Wir löschten, die von Simulink erstellte Variable $U$ aus dem Speicher. Diese soll die Eingangsvariable aus dem Workspace nun nicht mehr überschreiben können, sodass der richtige Wert übernommen wird. \\
So haben wir es dann geschafft nach einiger Zeit, durch Analysieren, Recherchieren und Ausprobieren, Matlab Simulink mit dem Workspace zu verknüpfen und somit das Projekt erfolgreich durchzuführen. Das Installieren einer anderen Matlab Version war auch nicht mehr notwendig. \\