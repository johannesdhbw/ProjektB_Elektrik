%Fazit - Autor:
\label{Fazit}

Abschließend kann man sagen, dass es beim Händling mit Matlab einige Probleme aufgrund von fehlenden Kenntnissen und Erfahrungen gab. So war uns anfangs nicht bewusst, dass die Eingangsvariablen bei Matlab Simulink händisch mit einer Referenz auf die im Workspace deklarierten Variablen zu versehen sind. Nachdem wir das festgestellt und behoben haben, wurden die Plots zwar verändert, jedoch entsprachen sie nicht unseren Erwartungen. \\
Eine Überlegung war, die Eingangsvariable händisch innerhalb in Matlab Simulink zu verändern, sodass wir dann das richtige Ergebnis bezüglich der Graphen herausbekommen. Jedoch hätte dies nach unserer Bewertung das Ziel, den Sinn und damit das ganze Projektergebnis verfälscht, sodass wir uns darauf festgelegt haben diese Möglichkeit nicht in Betracht zu ziehen. \\
Nach weiterer Auseinandersetzung und Analyse unseres Programmes und den Aufbau des Quellcodes im Workspace und der Schaltung in Simulink, haben wir festgestellt, dass aufgrund der Historie von Referenzen unserer Variablen, Matlab die Eingangsvariable $U$ zwar verändert, jedoch dann einmalig abgespeichert hat, sodass Veränderungen nicht übernommen wurden, da die von Simulink erstellte Variable, die aus dem Workspace überschreibt. Daher mussten wir uns die Model Explorer erstellten Variablen und deren Referenzen genauer anschauen und löschten, die von Simulink erstellte Variable $U$ aus dem Speicher, mit dem Hintergedanken, dass diese die Eingangsvariable nicht mehr überschreiben kann und somit der richtige Wert aus dem Workspace übernommen wird. So haben wir es dann geschafft nach einiger Zeit, durch Analysieren, Recherchieren und Ausprobieren, Matlab Simulink mit dem Workspace zu verknüpfen und somit das Projekt erfolgreich durchzuführen. \\