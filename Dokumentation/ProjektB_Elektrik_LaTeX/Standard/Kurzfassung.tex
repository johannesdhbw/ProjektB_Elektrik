% Kurzfassung - Autor:
\label{kurzfassung}

Der folgende Bericht wurde von zwei Studierenden der \ac{DHBW}-Ravensburg am Campus Friedrichshafen verfasst. Im Rahmen des Moduls Simulationstechnik sollen die Studierenden mithilfe von Matlab und Matlab-Simulink das Verhalten des Gleichstrommotors eines Unwuchtsystems vereinfachen, simulieren und die darauf folgenden Ergebnisse analysieren und interpretieren. \\
Zunächst wird das Originalsystem analysiert. Dieses wird zur genaueren Betrachtung in kleine Teilsysteme unterteilt. In diesem Fall wird der Gleichstromkreis des Elektromotors mithilfe eines physikalischen Modells dargestellt. Dies soll den Grundstein für die spätere Simulation legen. \\
Als nächstes wird das mathematische Modell aufgestellt, in welchem die Spannung die einzige Eingangsgröße darstellt. Anschließend werden Systemgleichungen aufgestellt, in welchem die Zusammenhänge der Ein- und Ausgangsgrößen klar werden. Zu guter Letzt sollen die aufgestellten Modelle in Matlab-Simulink mithilfe von Blockdiagrammen implementiert werden. \\
Im folgenden Bericht werden alle Modelle dargestellt, erläutert und analysiert.