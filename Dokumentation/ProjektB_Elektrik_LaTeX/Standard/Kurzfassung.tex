% Kurzfassung - Autor:
\label{kurzfassung}

Der folgende Bericht wurde von zwei Studierenden der \ac{DHBW}-Ravensburg am Campus Friedrichshafen verfasst. Im Rahmen des Moduls Simulationstechnik sollen die Studierenden mithilfe von Matlab und Matlab-Simulink das Verhalten des Gleichstrommotors eines Unwuchtsystems vereinfachen, simulieren und die darauf folgenden Ergebnisse analysieren und interpretieren.

Dafür beginnt man mit der Analyse des Originalsystems, wobei dieses in kleinere Teilsysteme, zur genaueren Betrachtung, unterteilt wird. In diesem Fall wird der Gleichstromkreis des Elektromotors mit Hilfe eines physikalischen Modells dargestellt, welches den Grundstein für die spätere Simulation bietet. Denn das Antriebsmoment entsteht in Abhängigkeit der an den Gleichstromkreis angelegten Spannung. Im nächsten Schritt wird das mathematische Modell aufgestellt, dabei ist die einzige Eingangsgröße, wie aus dem physikalischem Modell hervorgeht, die Spannung U. Ausgangsgrößen sind Ma, Fu, omega und s, für die Berechnung dieser ist es nötig die jeweiligen Systemgleichungen aufzustellen. Sind die Modelle zur Vereinfachung erstellt geht es daran, diese in Matlab-Simulink zu Übertragen und ein funktionierendes Blockdiagramm zu modellieren. Ist die Implementierung erfolgreich gewesen wird der Parametersatz zur Simulation in Matlab programmiert. Hier ist es möglich durch Variation der Parameter verschiedene Einwirkungen von außen zu simulieren.